\documentclass{article}
\usepackage[utf8]{inputenc}
\usepackage[T1]{fontenc}
\usepackage[a4paper, margin=1in]{geometry}
\usepackage[spanish]{babel}
\usepackage{gnuplottex}
\usepackage{fancyhdr}
\usepackage{minted}
\usepackage{multicol}
\usepackage{algpseudocode}
\usepackage{hyperref}
\usepackage{tikz}
\usetikzlibrary{arrows.meta}
\tikzset{>=stealth}
\tikzset{vertex/.style = {shape=circle,draw,minimum size=2em}}
\tikzset{edge/.style = {-Latex}}

\title{Implementación en C++ del Algoritmo de Dijkstra sobre montículo sesgado}
\author{David Davó Laviña}
\date{\today{}}

\pagestyle{fancy}
\fancyhf{}
\makeatletter
\fancyhead[RO]{\@author}
\fancyhead[LO]{\@title}
\makeatother

\setminted{
	frame=lines,
	baselinestretch=1.2,
}
\usemintedstyle {emacs}

\begin{document}
\section{Decrecer clave en montículos autoajustables}
Para poder implementar de forma eficiente decrecerClave, necesitamos conocer la
posición del nodo en el montículo, pues no podemos implementar la búsqueda en coste
menor que $\mathcal{O}(n)$ si no están ordenados los elementos. Se ha implementado
decrecerClave siguiendo la idea de los \textit{pairing heaps} \cite{PairingHeap}. Si el nodo
a eliminar no es la raíz del árbol, cortamos la arista que une el nodo con su padre y
enlazamos los dos árboles formados por el corte. Para ello ha sido necesario
implementar un tercer puntero en los nodos (\texttt{n->\_up}) con una referencia al padre
que será actualizado en las funciones que modifiquen el montículo.

El coste de unir dos montículos sesgados es $\mathcal{O}(\log n)$, siendo n el número de elementos en los montículos involucrados \cite{SkewHeap}. Cortar la arista que une al nodo con su padre podemos hacerlo en $\mathcal{O}(1)$, por lo que el coste de decrecerClave será el mismo que el de unir. 
% La única duda que nos queda es: ¿Seguirá siendo un montículo sesgado tras quitarse ese nodo y sus hijos?. El nodo quitado sí es un montículo sesgado, pues 

\section{Monticulos sesgados en C++}
Para el montículo se ha definido una clase parametrizada \mintinline{C++}|SkewHeap<K,V>|, donde \texttt{K} es el tipo de la clave y \texttt{V} el del valor a asignar. Tanto en los tests como en el análisis estadístico se ha usado la clase con claves de tipo \mintinline{C++}|unsigned|. Dicha clase guarda tan solo una variable de tipo \textit{Node}, que es la raíz del montículo. La clase \textit{Nodo} guarda la clave, el valor, y 3 punteros: a su padre, su hijo izquierdo, y su hijo derecho.

Las dos funciones más relevantes, \textit{join} y \textit{decreaseKey}, se han implementado tal que así:

\begin{minted}[linenos]{C++}
Node * join(Node * n1, Node * n2) {
  if (n1 == nullptr) return n2;
  if (n2 == nullptr) return n1;

  if (n1->_key > n2->_key)
	std::swap(n1, n2);

  std::swap(n1->_left, n1->_right);
  n1->_left = join(n1->_left, n2);
  n1->_left->_up = n1;

  return n1;
}
\end{minted}
\begin{minted}[linenos]{C++}
void decreaseKey(Node * node, K newKey) {
  if (_root == nullptr) throw EmptyHeapException();
  if (node == nullptr) throw std::invalid_argument("Can't decrease nullptr");
  if (newKey > node->_key) throw KeyGreaterException();

  auto & up = node->_up;
  node->_key = newKey;

  if (up != nullptr) {
	if (node != up->_right)
	  std::swap(up->_left, up->_right);

	up->_right = nullptr;
	node->_up = nullptr;

	_root = join(_root, node);
	_root->_up = nullptr;
}
\end{minted}

\section{Representación de los grafos en C++}
El grafo se ha representado como una lista de adyaciencia parametrizada
en el tipo de los pesos de las aristas, implementada como un vector estándar del 
tipo \texttt{Vertex<W>}. El tipo Vertex se ha implementado como un struct que
contiene una \texttt{string} etiqueta (no usada en el algoritmo), un descriptor de
vertice (\texttt{vertex\_descriptor}) que le identifica en el grafo (su posición en
la lista de adyacencia), y un vector estándar de aristas (\texttt{Edge<W>}). Cada arista es un struct compuesto por el descriptor de arista (\texttt{edge\_descriptor}), el peso de tipo \texttt{W}, y los descriptores de vértice tanto del vertice al que va, como del que viene. El descriptor de vértice del vértice que viene junto al descriptor de arista nos permiten identificar una arista en el grafo. El vértice con descriptor de vértice 0 se considerará el vértice nulo, cualquier intento de usar la función para añadir una arista al vértice nulo resultará en excepción.
\begin{minted}{C++}
typedef unsigned edge_descriptor;
typedef unsigned vertex_descriptor;

template <class W> 
struct Edge {
    edge_descriptor ed;
    vertex_descriptor to;
    vertex_descriptor from;
    W weight;
};

template <class W>
struct Vertex {
    vertex_descriptor vd;
    std::string label;
    std::vector<Edge<W>> edges;
};

template <class W> using Graph = std::vector<Vertex<W>>;
\end{minted}

\section{Algoritmo de Dijkstra en C++}
Nuestra función recibe un grafo y un vértice "fuente" desde el que calcularemos los caminos mínimos. Retorna un vector de pares donde, para cada uno de los vértices, retorna la suma de los pesos por ese camino, y el antecesor a ese nodo, produciendo el árbol de caminos mínimos. Para los vértices no conectados el peso del camino será el máximo que permita el tipo, y el predecesor será el vértice nulo (vértice con descriptor 0).

Por conveniencia declaramos dos vectores distintos, el vector de caminos y el vector de distancias. También declaramos la cola de prioridad implementada con el montículo sesgado y una lista de referencias a los nodos, para poder asociar cada vértice con su entrada en la lista de prioridad y poder decrecer la clave.

Primero poblamos las tablas de distancias (con 0 si es el vértice fuente y $\infty$\footnote{\mintinline{C++}|std::numeric_limits<W>::max()|} en caso contrario) y camino (con el vértice nulo como predecesor). Además, insertamos los nodos en la cola de prioridad y guardamos sus referencias en la tabla para poder decrecer el valor de la clave en el montículo. Hay que tener cuidado con el tipo de los pesos, pues no hay ningún tipo de detección de desbordamiento.

A continuación, realizamos el algoritmo de Dijkstra: mientras que la cola no esté vacía, sacamos el nodo más cercano y comprobamos todos sus vecinos (no comprobados anteriormente), actualizando la lista de distancias, caminos y la cola de prioridad si encontramos un camino más corto que el que teníamos anteriormente. Como durante el bucle se cumple el invariante $Q = V - S$ \cite{CormenDijsktra}, siendo $Q$ la cola de prioridad, $V$ la lista de vértices y $S$ los nodos ya visitados, podemos ver los nodos que hemos visitado usando solo la lista de referencias a la cola de prioridad, si tras visitar un nodo modificamos su entrada en la lista de referencias a nullptr.

Finalmente, retornamos el \textit{zip} de ambos vectores.

\begin{minted}{C++}
template <class W>
std::vector<std::pair<W, vertex_descriptor>> dijkstra(
	const Graph<W> & g, vertex_descriptor vd) {
  if (vd == 0) throw std::range_error("vertex can't be 0 (NULL)");
  std::vector<W> distances(g.size());
  std::vector<vertex_descriptor> path(g.size());

  SkewHeap<W, vertex_descriptor> pq;
  using SHNode = typename SkewHeap<W, vertex_descriptor>::Node;
  std::vector<SHNode *> nodes(g.size());

  for (vertex_descriptor i = 1; i < g.size(); ++i) {
    distances[i] = (i==vd)?0:std::numeric_limits<W>::max();
    path[i] = 0;
    nodes[i] = pq.insert(distances[i], i);
  }

  while(!pq.empty()) {
    vertex_descriptor currentV = pq.getMin()->getVal();
    nodes[pq.getMin()->getVal()] = nullptr;
    pq.deleteMin();

    for (Edge<W> e : g[currentV].edges) {
      if (nodes[e.to] == nullptr) continue;
      W tmpdst = distances[currentV] + e.weight;
      if (tmpdst < distances[e.to]) {
        distances[e.to] = tmpdst;
        path[e.to] = currentV;
        pq.decreaseKey(nodes[e.to], tmpdst);
      }
    }
  }

  /**
  ...
  return zip(distances, path)
  **/
}
\end{minted}

\section{Tests Unitarios en C++}
Para comprobar que se han implementado correctamente las estructuras de datos y los algoritmos, se han realizado distintos tests unitarios con Googletest\footnote{\url{https://github.com/google/googletest}}.

Como casos de prueba automatizados para el algoritmo de Dijsktra se han usado los dos siguientes grafos:

\vspace*{1em}
\noindent\begin{minipage}{.45\textwidth}
\centering
\begin{tikzpicture}
\node[vertex] (v1) at (0,0) {1};
\node[vertex] (v2) at (2,0) {2};

\draw[edge] (v1) to node[above] {100} (v2);
\end{tikzpicture}
\end{minipage} %
\begin{minipage}{.45\textwidth}
\centering
\begin{tikzpicture}
\node[vertex] (v1) at (0,0) {1};
\node[vertex] (v2) at (2,1) {2};
\node[vertex] (v3) at (2,-1) {3};

\draw[edge] (v1) to node[above] {500} (v2);
\draw[edge] (v1) to node[below] {100} (v3);
\draw[edge] (v3) to node[right] {50} (v2);
\end{tikzpicture}
\end{minipage}

El primero es trivial, y el segundo debería devolver el camino 1, 3, 2 con peso 150 para el nodo 2; y el camino 1,3 para el nodo 3; el nodo 1 debería tener como predecesor el vértice nulo y coste 0. También se han realizado pruebas con grafos más grandes ``a mano", como con el laberinto de Sedgewick \cite{SedgewickLabyrinth} y pequeños grafos generados aleatoriamente.

El código de dichos tests unitarios puede encontrarse en la carpeta \texttt{test/}

\section{Generación de grafos aleatorios}
Para poder tratar más fácilmente con distintos grafos se han creado funciones auxiliares que permitan leer archivos en formato \texttt{csv} y en el formato
\texttt{net}  usado por Pajek \cite{Pajek}, en el que solo es necesario especificar el numero de vértices y las aristas que los conectan (ocupando mucho menos espacio para grafos muy dispersos). Dichas funciones se han implementado en el fichero \texttt{graphReader.cpp}.

Para el tratamiento estadístico de datos se han generado grafos aleatorios con el modelo Erdős-Rényi, usando la función \texttt{gnp\_random\_graph} de la librería networkx\footnote{\url{https://networkx.github.io/}} de Python. Los grafos generados se pasarían mediante un pipe al programa \texttt{src/time.cpp}, que leería el grafo de la entrada estándar y retornaría el tiempo de ejecutar el algoritmo de Dijkstra, que será guardado en un archivo \texttt{csv} junto con el número de vértices y aristas del grafo generado para poder representar las gráficas de tiempo usando \texttt{gnuplot}. Se han generado diversos archivos con distinta densidad de vértices ($0.01$, $0.05$, $0.1$, $0.2$ y $0.5$). En lugar de guardar los archivos con los grafos generados (de hasta varios GB para una proporción de $0.5$), se han generado los grafos de prueba con la misma semilla: 42, por lo que los grafos se pueden volver a generar fácilmente usando la misma semilla y la misma función. Este método se ha ejecutado 3 veces y se ha calculado la media para cada uno de los 3 archivos generados con \textit{awk}.

\newpage
\section{Análisis de tiempo}
El insertar cada uno de los vértices a la cola de prioridad tendrá un coste de $\mathcal{O}(n\log n)$. A la hora de realizar el bucle mientras que no esté vacío, recorreremos todos los vértices eliminando la clave ($\mathcal{O} (\log n)$) y, para cada uno de sus vecinos, podemos decrecer la clave $\mathcal{O}(\log n)$. Esto nos deja un coste de $\mathcal{O}\left(a\log n+n\log n\right)$, donde $a$ es el número de aristas y $n$ el número de vértices. Podemos implementar el algoritmo en un tiempo menor con una cola de prioridad en la que \textit{decrecerClave} tenga coste constante, en cuyo caso el coste será $\mathcal{O}(a + n\log n)$. En nuestro caso, al generar grafos aleatorios con una densidad de aristas dada, el número de aristas será $a = \alpha\cdot(n\cdot(n-1))/2$, por lo que el coste en complejidad será del orden $\mathcal{O}((\alpha\cdot n^2+n)\log n)$. Cuando el grafo sea muy disperso ($\alpha$ pequeño), $\alpha\cdot n^2$ será menor que $n$. En la figura \ref{fig:graphTimeDijkstra} vemos el tiempo de ejecución en algoritmos aleatorios con distinto número de aristas (dependiente del número de vértices). Como podemos observar en la gráfica, se ajustan mucho a la función $(an^2+bn)\cdot log(n)$. Es más, la media de las desviaciones medias de las cuatro funciones es $\sigma=0.05486$

\begin{figure}
	\centering
	\begin{gnuplot}[terminal=epslatex, terminaloptions=color]
		xmax = 6000
		ymax = 20
		stons = 10**6
		set datafile separator comma
		
		set style line 1 pt 5 lc rgb "#e51e10"
		set style line 2 pt 5 lc rgb "#a009e73"
		set style line 3 pt 5 lc rgb "#e69f00"
		set style line 7 pt 5 lc rgb "#56b4e9"
				
		set xlabel "$|V|$"
		set ylabel "$t (ms)$"
		
		f01(x) = (a01*x**2+b01*x)*log(x)
		f05(x) = (a05*x**2+b05*x)*log(x)
		f10(x) = (a10*x**2+b10*x)*log(x)
		f20(x) = (a20*x**2+b20*x)*log(x)
	
		fit [0:xmax] [0:ymax] f01(x) "../out/gnp2_42_01.csv" u 1:($3/stons) via a01,b01
		fit [0:xmax] [0:ymax] f05(x) "../out/gnp2_42_05.csv" u 1:($3/stons) via a05,b05
		fit [0:xmax] [0:ymax] f10(x) "../out/gnp2_42_1.csv" u 1:($3/stons) via a10,b10
		fit [0:xmax] [0:ymax] f20(x) "../out/gnp2_42_2.csv" u 1:($3/stons) via a20,b20

		plot [0:xmax] [0:ymax] \
			f01(x) notitle w lines ls 1, \
			f05(x) notitle w lines ls 2, \
			f10(x) notitle w lines ls 3, \
			f20(x) notitle w lines ls 7, \
			"../out/gnp2_42_01.csv" u 1:($3/stons) w dots ls 1 notitle, \
			1/0 ls 1 lw 10 w p t "$\\alpha=0.01$", \
			"../out/gnp2_42_05.csv" u 1:($3/stons) w dots ls 2 notitle, \
			1/0 ls 2 lw 10 w p t "$\\alpha=0.05$", \
			"../out/gnp2_42_1.csv" u 1:($3/stons) w dots ls 3 notitle, \
			1/0 ls 3 lw 10 w p t "$\\alpha=0.10$", \
			"../out/gnp2_42_2.csv" u 1:($3/stons) w dots ls 7 notitle, \
			1/0 ls 7 lw 10 w p t "$\\alpha=0.20$"
	\end{gnuplot}
	\caption{Tiempo en milisegundos que tarda en ejecutarse el Algoritmo de Dijkstra con distintas densidades de nodos}
	\label{fig:graphTimeDijkstra}
\end{figure}

% \nocite{*} % <-- Recuerda quitar esto
\bibliographystyle{alpha}
\bibliography{references}

\end{document}